\documentclass[a4paper]{report}
\usepackage{titlesec}
\usepackage[margin=1.2in]{geometry}
\usepackage[backend=biber, style=authoryear, maxnames=999, maxcitenames=3, firstinits=true, urldate=long]{biblatex}

\addbibresource{citation.bib}

% start uwe referencing styling
\DeclareNameAlias{sortname}{last-first}
\DeclareFieldFormat{edition}{%
  \ifinteger{#1}
    {\ifnumequal{#1}{1}%
     {}%
     {\mkbibordedition{#1}~\bibstring{edition}}%
    }
    {#1\isdot}}

\DeclareFieldFormat[article,inbook,incollection]{title}{#1\isdot}
\DeclareFieldFormat[article,inbook,incollection]{citetitle}{#1\isdot}

\newrobustcmd{\MakeTitleCase}[1]{%
  \ifboolexpr{test {\ifentrytype{article}} or test {\ifentrytype{inbook}} or test {\ifentrytype{incollection}}}
    {#1}
    {\MakeSentenceCase{#1}}}

\DeclareFieldFormat{urldate}{\bibsentence\mkbibbrackets{\bibstring{urlseen}\space#1}}
\DeclareFieldFormat{url}{\bibstring{urlfrom}\addcolon\space\url{#1}}

\renewbibmacro*{journal}{%
  \iffieldundef{journaltitle}
    {}
    {\printtext[journaltitle]{%
       \printfield[titlecase]{journaltitle}%
       \setunit{\subtitlepunct}%
       \printfield[titlecase]{journalsubtitle}}
       \ifboolexpr{
         not test {\iffieldundef{url}}
         or
         not test {\iffieldundef{urldate}}
         or
         not test {\iffieldundef{doi}}
         or
         not test {\iffieldundef{eprint}}
       }
         {\nopunct\bibstring[\mkbibbrackets]{online}}%
         {}}}

\newbibmacro*{journal+issuetitle}{%
  \usebibmacro{journal}%
  \setunit*{\addspace}%
  \iffieldundef{series}
    {}
    {\newunit
     \printfield{series}%
     \setunit{\addspace}}%
  \newunit
  \usebibmacro{volume+number+eid}%
  \setunit{\addspace}%
  \usebibmacro{issue+date}%
  \setunit{\addcolon\space}%
  \usebibmacro{issue}%
  \newunit}

\NewBibliographyString{online}
\DefineBibliographyStrings{english}{%
  urlseen    = {accessed},
  online     = {online},
}
% end uwe referencing styling


\titleformat{\chapter}{\normalfont\huge}{\thechapter.}{20pt}{\huge\it}

\begin{document}

\title{Embedded Systems Design - Mobile Robot Research Platform}
\author{Ben Silvester}
\date{2022}
\maketitle

\tableofcontents

\newpage

\chapter{Task Brief}
The task involves creating an autonomous robot that follows a black line course on a white background. There are three tracks to complete:

\begin{enumerate}
	\item Simple loop track with left and right turns
	\item Figure of 8 track
	\item Crossings, tight corners and close parallel tracks
\end{enumerate}

Full marks only require the first. The final two are for competition purposes only, and involve a race focusing on the fastest successful completion of the tracks.

\hfill \break
The following is the basic function we must achieve:

\hfill \break
\noindent\fbox{%
    \parbox{\textwidth}{%
        Build an embedded controller that can navigate a predefined path on the floor marked as a black line on a white background.
    }%
}
\hfill \break

The following are the advanced functions desired:

\begin{itemize}
	\item Show visual indication of the robot's movements
	\item Show visual indication of the robot's speed
	\item Illuminate the track in darkness
	\item White lights on the front when powered
	\item Red brake lights on the rear for stopping or when not moving
	\item White lights on rear for reversing
	\item Yellow flashing indicators when cornering
\end{itemize}

\chapter{Approach}
The following section is split into various subsections after decomposing the problem.

\section{Line Sensing}
We are required to detect a black line on a white background. The width of the line is not known. There will also be a figure of 8 line, and crossings, tight corners and close parallel tracks. This section involves identifying possible methods for detecting the line, and differentiating the different junctions.

\subsection{IR Detector}
An IR (infra-red) LED and photodiode combination may be used to distinguish between black and white lines (\cite{adafruit}).

\printbibliography

\end{document}